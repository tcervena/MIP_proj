\documentclass[12pt]{article}
\title{Kritické faktory vplývajúce na úspešnosť e-learningu}
\author{Tatiana Červená}
\begin{document}
\maketitle
\section*{Abstrakt}
E-learning sa masívne využíva ako mocný prostriedok pri vzdelávaní, je však novou technológiou, a preto je dôležité vyhodnotiť všetky faktory, ktoré vplývajú na jeho efektivitu, na prácu a získavanie informácií z internetu a taktiež na upravenie štýlu výučby, aby sedela a dopĺňala sa s novými technológiami a platformami. V tejto práci sa budem venovať najdôležitejším faktorom, ktoré sú rozdelené do niekoľkých kategórii a venujú sa práve problematike e-learningu, potrebám študentov, vyučujúcich a aj technologickým prvkom využívaným pri výučbe, ktoré by mohli vplývať na konečný úspech študentov pri vzdelávaní, poprípade by mohli zjednodušiť adaptáciu z normálneho vzdelávania práve na to elektronické.
\section{Problematika faktorov}
Ako prvé, je podstatné vysvetliť, čo sú tieto kritické faktory a ako by mali vyzerať. Ako už aj samotné meno hovorí, budú to nejaké činitele, ktoré budeme vedieť spracovať do niekoľkých hlavných kategórii, pričom v nich môžeme nájsť aj iné menšie podkategórie, ktoré sa ďalej môžu rozvíjať a ísť do detailov, avšak nesmieme zabúdať na to, že tie už sú menej a menej rozhodujúce, čo sa týka úspešnosti e-learningu. Tieto faktory by sa mali dať nejakým spôsobom merať alebo skúmať a taktiež kontrolovať. V tomto článku si vyhradíme tri hlavné kategórie, ktorým sa budeme hlbšie venovať. Medzi ne patria technologické potreby, prístup pedagóga a školy a taktiež všeobecné potreby pre študenta. V článku sa nebudeme venovať prístupu študenta, zohľadníme síce jeho potreby a čo by mohlo vplývať na jeho prístup, ale vo všeobecnosti budeme rátať s tým, že je to človek, ktorý sa chce učiť a my iba hľadáme ako mu túto cestu zefektívniť.   
\subsection{Technologické potreby}
\subsubsection{Platformy}
Hlavnou súčasťou online štúdia je priestor, kde sa budeme vzdelávať. Kedže nie sme v triede, je potrebné, aby platfoma, ktorú využívame bola najmä prehľadná a intuitívna. Zároveň pri tom myslíme na všetky nástroje, ktoré pri výučbe budeme potrebovať. Program by mal byť jednoduchý, ale zároveň by mu nemali chýbať všetky nástroje, ktoré si daný predmet vyžaduje. Zároveň aj naopak, príliš veľa nepotrebných nástrojov by mohlo vplývať na študentov zbytočne mätúco.  

\end{document}