\documentclass[12pt]{article}
\usepackage{cite}
\title{Kritické faktory vplývajúce na úspešnosť e-learningu}
\author{Tatiana Červená}
\begin{document}
\maketitle
\section*{Abstrakt}
E-learning sa masívne využíva ako mocný prostriedok pri vzdelávaní, je však novou technológiou, a preto je dôležité vyhodnotiť všetky faktory, ktoré vplývajú na jeho efektivitu, na prácu a získavanie informácií z internetu a taktiež na upravenie štýlu výučby, aby sedela a dopĺňala sa s novými technológiami a platformami. V tejto práci sa budem venovať najdôležitejším faktorom, ktoré sú rozdelené do niekoľkých kategórii a venujú sa práve problematike e-learningu, potrebám študentov, vyučujúcich a aj technologickým prvkom využívaným pri výučbe, ktoré by mohli vplývať na konečný úspech študentov pri vzdelávaní, poprípade by mohli zjednodušiť adaptáciu z normálneho vzdelávania práve na to elektronické.
\section{Problematika faktorov}
Ako prvé, je podstatné vysvetliť, čo sú tieto kritické faktory a ako by mali vyzerať. Ako už aj samotné meno hovorí, budú to nejaké činitele, ktoré budeme vedieť spracovať do niekoľkých hlavných kategórii, pričom v nich môžeme nájsť aj iné menšie podkategórie, ktoré sa ďalej môžu rozvíjať a ísť do detailov, avšak nesmieme zabúdať na to, že tie už sú menej a menej rozhodujúce, čo sa týka úspešnosti e-learningu. Tieto faktory by sa mali dať nejakým spôsobom merať alebo skúmať a taktiež kontrolovať. V tomto článku si vyhradíme tri hlavné kategórie, ktorým sa budeme hlbšie venovať. Medzi ne patria technologické potreby, prístup pedagóga a školy a taktiež všeobecné potreby pre študenta. V článku sa nebudeme venovať prístupu študenta, zohľadníme síce jeho potreby a čo by mohlo vplývať na jeho prístup, ale vo všeobecnosti budeme rátať s tým, že je to človek, ktorý sa chce učiť a my iba hľadáme ako mu túto cestu zefektívniť.   
\subsection{Technologické potreby}
\subsubsection{Platformy}
Hlavnou súčasťou online štúdia je priestor, kde sa budeme vzdelávať. Kedže nie sme v triede, je potrebné, aby platfoma, ktorú využívame bola najmä prehľadná a intuitívna. Okrem toho môžeme prihliadať aj na jej spoľahlivosť, aby bola vždy študentom dostupná. Ako neskôr uvádza citovaný článok, jednoduchosť je kľúčová, keďže v prvom rade ide o pochopenie učiva a nie platformy, programu alebo iných technologických prvkov, ktoré majú štúdium iba zjednodušiť alebo sprostredkovať.   \cite{CSF1}
\subsubsection{Nástroje}  
Zároveň pri tom myslíme na všetky nástroje, ktoré pri výučbe budeme potrebovať. Program by mal byť jednoduchý, ale zároveň by mu nemali chýbať všetky nástroje, ktoré si daný predmet vyžaduje. Zároveň aj naopak, príliš veľa nepotrebných nástrojov by mohlo vplývať na študentov zbytočne mätúco. Platformy môžu byť buď vybraté vyučujúcim, ktorý by mal zodpovedne zvážiť, aká je vhodná pre jeho zámery.  
\subsection{prístup pedagóga a školy}
\subsubsection{Obsah výučby}
Práca pedagóga sa musí prispôsobiť podmienkam, čiže vysoko závisí na jeho zručnosti s novými technológiami. Zároveň však aj on musí prispôsobiť náplň a formu výučby. Kreatívna tvorba a nápady sú veľmi užitočné ako na nižších stupňoch, tak aj na vyšších, preto ich treba implementovať aj napríklad na vysokých školách. Tu je treba spomenúť napríklad pojem "gamification", ktorý sa dá veľmi dobre využiť pri online výučbe, pri ňom ide o používanie rôznych herných prvkov pri výučbe. Pri ňom je veľmi podstatná spätná väzba, keď študent môže ihneď vidieť svoj pokrok, napríklad pri rôznych cvičných testoch, poprípade úlohách, môžeme povedať,že vytvára akúsi ilúziu "levelov" a motivuje študentov. Zároveň môže ísť aj o rôzne zábavné aktivity a úlohy, no nesmieme zabúdať aby sme neporušili práve už spomenuté pravidlo o jednoduchosti a študent sa naozaj mohol sústrediť na náplň učiva.  \cite{gamif} 
\subsubsection{Podpora školy}
Pod podporou školy si predstavíme najmä "hmotné" benefity. Môžu sem napríklad patriť online knižnice alebo iné dokumentové servery s potrebnými podkladmi aj pre samoštúdium alebo pri tvorbe článkov a podobne. Zároveň štúdia uvádza, že pre študentov bol v tejto kategórii najkritickejší prístup k počítačom na univerzite. Ďalej tam patria aj iné prístroje ako napríklad tlačiareň, poprípade skener. \cite{2}
\subsection{Všeobecné potreby pre študenta}

\bibliographystyle{plain}
\bibliography{zdroje}
\end{document}